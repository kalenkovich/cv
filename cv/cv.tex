%!TEX TS-program = xelatex
%!TEX encoding = UTF-8 Unicode
% Awesome CV LaTeX Template for CV/Resume
%
% This template has been downloaded from:
% https://github.com/posquit0/Awesome-CV
%
% Author:
% Claud D. Park <posquit0.bj@gmail.com>
% http://www.posquit0.com
%
%
% Adapted to be an Rmarkdown template by Mitchell O'Hara-Wild
% 23 November 2018
%
% Template license:
% CC BY-SA 4.0 (https://creativecommons.org/licenses/by-sa/4.0/)
%
%-------------------------------------------------------------------------------
% CONFIGURATIONS
%-------------------------------------------------------------------------------
% A4 paper size by default, use 'letterpaper' for US letter
\documentclass[11pt, a4paper]{awesome-cv}

% Configure page margins with geometry
\geometry{left=1.4cm, top=.8cm, right=1.4cm, bottom=1.8cm, footskip=.5cm}

% Specify the location of the included fonts
\fontdir[fonts/]

% Color for highlights
% Awesome Colors: awesome-emerald, awesome-skyblue, awesome-red, awesome-pink, awesome-orange
%                 awesome-nephritis, awesome-concrete, awesome-darknight

\definecolor{awesome}{HTML}{000000}

% Colors for text
% Uncomment if you would like to specify your own color
% \definecolor{darktext}{HTML}{414141}
% \definecolor{text}{HTML}{333333}
% \definecolor{graytext}{HTML}{5D5D5D}
% \definecolor{lighttext}{HTML}{999999}

% Set false if you don't want to highlight section with awesome color
\setbool{acvSectionColorHighlight}{true}

% If you would like to change the social information separator from a pipe (|) to something else
\renewcommand{\acvHeaderSocialSep}{\quad\textbar\quad}

\def\endfirstpage{\newpage}

%-------------------------------------------------------------------------------
%	PERSONAL INFORMATION
%	Comment any of the lines below if they are not required
%-------------------------------------------------------------------------------
% Available options: circle|rectangle,edge/noedge,left/right

\name{Evgenii}{Kalenkovich}

\position{Junior Research Fellow}
\address{Institute for Cognitive Neuroscience, HSE University, Moscow,
Russia}

\mobile{+7 958 823 01 99}
\email{\href{mailto:e.kalenkovich@gmail.com}{\nolinkurl{e.kalenkovich@gmail.com}}}
\homepage{kalenkovich.netlify.app}
\github{kalenkovich}
\twitter{kalenkovich}

% \gitlab{gitlab-id}
% \stackoverflow{SO-id}{SO-name}
% \skype{skype-id}
% \reddit{reddit-id}

\quote{I am an informal pre-doc on the lookout for a PhD project,
preferably at the intersection of Cognitive Modeling, Psycholinguistics,
and Developmental Psychology. Meanwhile, I am working on three
psycho-/neurolinguistic projects: a study of syntax processing using
MEG, a study of negation processing using mouse tracking, and a study of
phoneme perception using selective adaptation. In my spare time, I am
involved in several other mini-projects: development of a Python package
to analyze frequency-tagging data, unsupervised classification of MEG
data for a BIOMAG 2022 competition, writing a tutorial on the use of
workflow management systems in M/EEG data analysis, reproducing early
PDP models (such as interactive activation model). I do my current best
to work with computational reproducibility and open science practices in
mind. Here and there I make small contributions to scientific
open-source projects.}

\usepackage{booktabs}

\providecommand{\tightlist}{%
	\setlength{\itemsep}{0pt}\setlength{\parskip}{0pt}}

%------------------------------------------------------------------------------



% Pandoc CSL macros
\newlength{\cslhangindent}
\setlength{\cslhangindent}{1.5em}
\newlength{\csllabelwidth}
\setlength{\csllabelwidth}{3em}
\newenvironment{CSLReferences}[3] % #1 hanging-ident, #2 entry spacing
 {% don't indent paragraphs
  \setlength{\parindent}{0pt}
  % turn on hanging indent if param 1 is 1
  \ifodd #1 \everypar{\setlength{\hangindent}{\cslhangindent}}\ignorespaces\fi
  % set entry spacing
  \ifnum #2 > 0
  \setlength{\parskip}{#2\baselineskip}
  \fi
 }%
 {}
\usepackage{calc}
\newcommand{\CSLBlock}[1]{#1\hfill\break}
\newcommand{\CSLLeftMargin}[1]{\parbox[t]{\csllabelwidth}{#1}}
\newcommand{\CSLRightInline}[1]{\parbox[t]{\linewidth - \csllabelwidth}{#1}}
\newcommand{\CSLIndent}[1]{\hspace{\cslhangindent}#1}

\begin{document}

% Print the header with above personal informations
% Give optional argument to change alignment(C: center, L: left, R: right)
\makecvheader

% Print the footer with 3 arguments(<left>, <center>, <right>)
% Leave any of these blank if they are not needed
% 2019-02-14 Chris Umphlett - add flexibility to the document name in footer, rather than have it be static Curriculum Vitae
\makecvfooter
  {June 2021}
    {Evgenii Kalenkovich~~~·~~~Curriculum Vitae}
  {\thepage}


%-------------------------------------------------------------------------------
%	CV/RESUME CONTENT
%	Each section is imported separately, open each file in turn to modify content
%------------------------------------------------------------------------------



\hypertarget{experience-briefly-see-jobs-for-details}{%
\section{Experience briefly (see ``Jobs'' for
details)}\label{experience-briefly-see-jobs-for-details}}

\begin{cventries}
    \cventry{Various positions, at times without official affiliation}{HSE University}{Moscow}{April 2015--present}{\begin{cvitems}
\item Helping with Matlab scripts to analyze EEG data
\item Writing software for real-time EEG source reconstruction and 3D visualization
\item Running a psycholinguistic mouse-tracking study of negation processing
\item Running a neurolinguistic MEG study of syntax processing
\item Teaching Digital Signal Processing, Experimental Design, Statistics (as a discussion/recitation session teacher)
\item Getting an MSc in 'Cogntive Science and Technologies' with an EEG source reconstruction project modeling surface propagation of epileptic spikes
\end{cvitems}}
    \cventry{Maths Tutor}{Self-employed}{Moscow}{September 2014--December 2015}{\begin{cvitems}
\item Helping senior high school students excel at a standardized maths test to get into a good university
\item Helping senior high school students pass a standardized maths test to graduate at all
\item Helping maths-challenged universtiy students understand enough calculus or linear algebra to pass
\end{cvitems}}
    \cventry{Credit Risk Analyst}{Large Russian Banks}{Moscow}{July 2011--July 2014}{\begin{cvitems}
\item Credit risk models, credit portfolio analysis, lots of SQL, Excel, and Powerpoint
\end{cvitems}}
\end{cventries}

\hypertarget{skills}{%
\section{Skills}\label{skills}}

\begin{cventries}
    \cventry{data analysis, statistical analysis (NHST/Bayesian), computationally reproducible research, MEEG data analysis (sensor/time-frequency/sources), sample size assessment (NHST/Bayesian), not scared of Mathematics}{Analytical}{}{}{}\vspace{-4.0mm}
    \cventry{Python, R, Matlab, custom online experiment building (Django/HTML/css/JavaScript)", SQL, Bash}{Programming}{}{}{}\vspace{-4.0mm}
    \cventry{tidyverse, lme4, simr/faux/BFDA (simulation-based sample size assessment), papaja, mne-python, psychopy, scipy/numpy/pandas/matplotlib, PyQT}{Packages}{}{}{}\vspace{-4.0mm}
    \cventry{conda, renv, Snakemake, Git, Django}{Tools}{}{}{}\vspace{-4.0mm}
\end{cventries}

\hypertarget{education}{%
\section{Education}\label{education}}

\begin{cventries}
    \cventry{MSc in Cognitive Science}{HSE University, 
School of Psychology}{Moscow}{2015 - 2017}{}\vspace{-4.0mm}
    \cventry{MSc in Financial and Actuarial Mathematics}{Charles University, 
Faculty of Mathematics and Physics}{Prague}{2008 - 2010}{}\vspace{-4.0mm}
    \cventry{BSc in Financial Mathematics}{Charles University, 
Faculty of Mathematics and Physics}{Prague}{2005 - 2008}{}\vspace{-4.0mm}
\end{cventries}

\pagebreak

\hypertarget{publications}{%
\section{Publications}\label{publications}}

\hypertarget{published}{%
\subsection{Published}\label{published}}

\begin{cventries}
    \cventry{Yuri G. Pavlov, Nika Adamian, Stefan Appelhoff, Mahnaz Arvaneh, Christopher S. Y. Benwell, Christian Beste, Amy R. Bland, Daniel E. Bradford, Florian Bublatzky, Niko A. Busch, Peter E. Clayson, Damian Cruse, Artur Czeszumski, Anna Dreber, Guillaume Dumas, Benedikt Ehinger, Ganis Giorgio, Xun He, José A. Hinojosa, Christoph Huber-Huber, Michael Inzlicht, Bradley N. Jack, Magnus Johannesson, Rhiannon Jones,\textbf{ Evgenii Kalenkovich}, Laura Kaltwasser, Hamid Karimi-Rouzbahani, Andreas Keil, Peter König, Layla Kouara, Louisa Kulke, Cecile D. Ladouceur, Nicolas Langer, Heinrich R. Liesefeld, David Luque, Annmarie MacNamara, Liad Mudrik, Muthuraman Muthuraman, Lauren B. Neal, Gustav Nilsonne, Guiomar Niso, Sebastian Ocklenburg, Robert Oostenveld, Cyril R. Pernet, Gilles Pourtois, Manuela Ruzzoli, Sarah M. Sass, Alexandre Schaefer, Magdalena Senderecka, Joel S. Snyder, Christian K. Tamnes, Emmanuelle Tognoli, van Marieke K. Vugt, Edelyn Verona, Robin Vloeberghs, Dominik Welke, Jan R. Wessel, Ilya Zakharov, Faisal Mushtaq}{\#EEGManyLabs: Investigating the Replicability of Influential EEG Experiments}{Cortex}{2021}{}\vspace{-4.0mm}
    \cventry{Nikita A. Novikov, Yulia M. Nurislamova, Natalia A. Zhozhikashvili,\textbf{ Evgenii E. Kalenkovich}, Anna A. Lapina, Boris V. Chernyshev}{Slow and Fast Responses: Two Mechanisms of Trial Outcome Processing Revealed by EEG Oscillations}{Frontiers in Human Neuroscience}{2017}{}\vspace{-4.0mm}
\end{cventries}

\hypertarget{in-progress}{%
\subsection{In progress}\label{in-progress}}

\begin{cventries}
    \cventry{\textbf{Evgenii Kalenkovich}, Anna Shestakova, Nina Kazanina}{Frequency Tagging of Syntactic Structure or Lexical Properties; A Registered MEG Study}{}{Stage 1 of Registered Report received In-Principle Acceptance}{}\vspace{-4.0mm}
    \cventry{\textbf{Evgenii Kalenkovich}, Ekaterina Stupina}{Phonemes in Speech Perception: Selective Adaptation to Missing Phonemes}{}{currently being revised}{}\vspace{-4.0mm}
\end{cventries}

\hypertarget{jobs}{%
\section{Jobs}\label{jobs}}

\hypertarget{academic}{%
\subsection{Academic}\label{academic}}

\begin{cventries}
    \cventry{Junior Research Fellow}{Institute for Cognitive Neuroscience, HSE University}{Moscow}{January 2019--present}{\begin{cvitems}
\item Running psycho/neurolinguistics projects under the supervision of Dr. Nina Kazanina
\end{cvitems}}
    \cventry{Research Assistant}{Institute for Cognitive Neuroscience, HSE University}{Moscow}{July 2018--December}{\begin{cvitems}
\item Running psycho/neurolinguistics projects under the supervision of Dr. Nina Kazanina
\end{cvitems}}
    \cventry{Not Officiall Employed}{Institute for Cognitive Neuroscience, HSE University}{Moscow}{January 2018--June 2018}{\begin{cvitems}
\item Running psycho/neurolinguistics projects under the supervision of Dr. Nina Kazanina
\end{cvitems}}
    \cventry{Research Assistant}{Centre for Bioelectric Interfaces, HSE University}{Moscow}{July 2017--March 2018}{\begin{cvitems}
\item Developing real-time EEG source reconstruction and 3D visualization software (PyQt, mne-python, OpenGL)
\end{cvitems}}
    \cventry{Research Programmer}{Cognitive Psychophysiology, HSE University}{Moscow}{April 2015--April 2016}{\begin{cvitems}
\item Writing Matlab scripts for EEG time-frequency analysis
\end{cvitems}}
\end{cventries}

\hypertarget{non-academic}{%
\subsection{Non-academic}\label{non-academic}}

\begin{cventries}
    \cventry{Chief Analyst}{Risk Management Division, OTP Bank}{Moscow}{January 2014--July 2014}{\begin{cvitems}
\item Debt collection analysis and modelling
\end{cvitems}}
    \cventry{Chief Analyst}{Credit Scoring Unit, Russian Standard Bank}{Moscow}{July 2013--December 2013}{\begin{cvitems}
\item Credit scoring models for potential clients, credit risk analysis of current clients
\end{cvitems}}
    \cventry{Lead Analyst}{Credit Scoring Unit, Russian Standard Bank}{Moscow}{July 2011--June 2013}{\begin{cvitems}
\item Credit risk analysis of current clients
\end{cvitems}}
\end{cventries}

\hypertarget{posters}{%
\section{Posters}\label{posters}}

\begin{cventries}
    \cventry{\textbf{Evgenii Kalenkovich}, Nina Kazanina}{Syntactic vs. Lexical Effects in Syntactic Frequency Tagging}{SNL 2020, Online}{October 2020}{}\vspace{-4.0mm}
    \cventry{\textbf{Evgenii Kalenkovich}, Emily J. Darley, Christopher Kent, Nina Kazanina}{Psychological mechanisms contributing to the effort in negation processing}{AMLaP 2020, Online}{September 2020}{}\vspace{-4.0mm}
    \cventry{\textbf{Evgenii Kalenkovich}, Emily J. Darley, Christopher Kent, Nina Kazanina}{Competition between local priming and global outcome in processing of negatives}{AMLaP 2019, Moscow}{September 2019}{}\vspace{-4.0mm}
    \cventry{\textbf{Evgenii Kalenkovich}, Ekaterina Stupina}{Do we need phonemes in speech perception? An auditory selective adaptation study}{AMLaP 2019, Moscow}{September 2019}{}\vspace{-4.0mm}
\end{cventries}

\hypertarget{training}{%
\section{Training}\label{training}}

\hypertarget{summer-schools}{%
\subsection{Summer schools}\label{summer-schools}}

\begin{cventries}
    \cventry{University of Groningen, Groningen}{\href{https://www.rug.nl/research/fse/cognitive-systems-and-materials/news/seminars/agenda/2019/20190408_springschoolcognitivemodeling?lang=en}{Groningen Spring School on Cognitive Modeling}}{2019}{}{\begin{cvitems}
\item Error-driven learning: simulating the time course of learning (5 days, active participation)
\item Nengo: converting high-level cognitive theories into low-level spiking neuron implementations (5 days, auditing)
\end{cvitems}}
    \cventry{Psychology Institute of Russain Academy of Sciences, Moscow}{Network modeling and analysis in human cognition research school}{2018}{}{\begin{cvitems}
\item Basics of graph theory: measures, statistics, application (4 days, last day - group project appying network analysis to the Remote Assocation Task (RAT) data)
\end{cvitems}}
\end{cventries}

\hypertarget{moocs}{%
\subsection{MOOCs}\label{moocs}}

\begin{cventries}
    \cventry{Coursera, online}{\href{https://www.coursera.org/learn/datasciencemathskills}{Data Science Math Skills}}{2020}{}{\begin{cvitems}
\item This course covered high-school maths. Honestly, this was an attempt at productive procrastination during the COVID-19 lockdown. \href{https://coursera.org/share/d06128df3dbf0ed99de987da0d257f09}{Certificate.}
\end{cvitems}}
    \cventry{Coursera, online}{\href{https://www.coursera.org/learn/designexperiments}{Designing, Running, and Analyzing Experiments}}{2018}{}{\begin{cvitems}
\item \href{https://coursera.org/share/13d8627b09f871715aec86dcd7dd0fbe}{Certificate.}
\end{cvitems}}
    \cventry{Coursera, online}{\href{https://www.coursera.org/learn/discrete-calculus}{Single Variable Calculus (5 courses)}}{2018}{}{\begin{cvitems}
\item In addition to being a refresher of Calculus 101, this course taught me discrete approximation for differential equation solutions. Completed, no certificate.
\end{cvitems}}
    \cventry{Coursera, online}{\href{https://www.coursera.org/specializations/data-science-python}{Applied Data Science with Python specialization}}{2017}{}{\begin{cvitems}
\item Five courses covering basics of data manipulation, text mining, machine learning, social network analysis, plotting and data representation. \href{https://coursera.org/share/cb4c8abff922cf688a98cc30d55cff1a}{Certificate.}
\end{cvitems}}
    \cventry{Udacity, online}{\href{https://classroom.udacity.com/courses/cs101}{Intro to Computer Science}}{2014}{}{\begin{cvitems}
\item An introductory Python course. \href{https://confirm.udacity.com/QXWAQCFL}{Certificate.}
\end{cvitems}}
    \cventry{Coursera, online}{Computing for Data Analysis}{2013}{}{\begin{cvitems}
\item 'In this course students learn programming in R, reading data into R, creating data graphics, accessing and installing R packages, writing R functions, debugging, and organizing and commenting R code.' \href{https://www.coursera.org/api/legacyCertificates.v1/spark/statementOfAccomplishment/970940~3547767/pdf}{Certificate.}
\end{cvitems}}
    \cventry{Udacity, online}{\href{https://classroom.udacity.com/courses/cs253}{Web Development}}{2013}{}{\begin{cvitems}
\item Basics, forms, templates, databases, user accounts, security, APIs, caching, scaling. \href{https://confirm.udacity.com/7H2KDEWK}{Certificate.}
\end{cvitems}}
\end{cventries}

\hypertarget{teaching}{%
\section{Teaching}\label{teaching}}

\hypertarget{university-courses}{%
\subsection{University courses}\label{university-courses}}

\begin{cventries}
    \cventry{Psychology Department}{Python for Data Extraction and Data Mining}{HSE University, Moscow}{April-June, 2020}{\begin{cvitems}
\item Introduction to Python for first-year bachelor students. The course blended a MOOC with 5 remote discussion/practice sessions. I led the sessions for four groups of students out of seven.
\end{cvitems}}
    \cventry{University-wide course}{Introduction to Programming}{HSE University, Moscow}{September-December, 2019 and 2020}{\begin{cvitems}
\item Introduction to Python for third/fourth-year bachelor students from various departments. The course blended a MOOC with ~7 lectures and ~14 discussion practice sessions. I led the session for two groups of students.
\end{cvitems}}
    \cventry{Master's program 'Cognitive science and technologies'}{Qualitative and Quantitative Research Methods in Psychology}{HSE University, Moscow}{September, 2018 - March, 2019}{\begin{cvitems}
\item From the syllabus: 'The course reviews the principal steps taken during a psychological research study and aims to provide students with the knowledge and competencies necessary to plan and conduct research projects of their own leading to M.Sc. dissertation and future scientific publications.' I led the discussion/practice sessions for the first-year cognitive science Master's students. The sessions combined the material from the lectures with a pretty much independent course on statistics using R.
\end{cvitems}}
    \cventry{Master's program 'Cognitive science and technologies'}{Digital Signal Processing}{HSE University, Moscow}{January-March, 2017}{\begin{cvitems}
\item Basics of digital signal processing: Fourier, linear filters, random processes. I led approximately seven discussion/practice sessions.
\end{cvitems}}
\end{cventries}

\hypertarget{mini-courses}{%
\subsection{Mini courses}\label{mini-courses}}

\begin{cventries}
    \cventry{Applied Cogntive Science Summer School}{Reproducible Resarch}{HSE University, Moscow}{November, 3-5, 2020}{\begin{cvitems}
\item Three 2-hour-long workshops on reproducible code using git, reproducible software environment using conda, reproducible articles using renv, r-markdown, and papaja.
\end{cvitems}}
    \cventry{Applied Cogntive Science Summer School}{Component Analysis and Dimensionality Reduction for M/EEG data}{HSE University, Moscow}{June, 15-18, 2019}{\begin{cvitems}
\item Three 2-hour-long workshops taught together with Egor Levchenko. Intuitive visual explanation of covariance matrices, whitening and decorrelation methods in general. A short detour into ICA. Examples of applications from authors' research: inter-subject correlation of MEG data acquired during naturalistic stimuli viewing and frequency-tagging analysis of MEG data acquired during listening to speech.
\end{cvitems}}
\end{cventries}

\end{document}
